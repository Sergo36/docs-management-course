\documentclass{article}

% Подключение пакетов
\usepackage[T2A]{fontenc}
\usepackage[utf8]{inputenc}
\usepackage[english,russian]{babel}
\usepackage{graphicx}
\usepackage{float}
\usepackage{hyperref}

% Настройка гиперссылок
\hypersetup{
    colorlinks=true,
    linkcolor=blue,
    filecolor=magenta,      
    urlcolor=cyan,
}

% Начало документа
\begin{document}

% Титульный лист
\title{Шаблон процесса реализации по предмету "Управление проектной документацией"}
\author{Ваше имя}
\date{\today}
\maketitle

% Оглавление
\tableofcontents
\newpage

% Введение
\section{Введение}
Этот шаблон предоставляет общую структуру для процесса реализации по предмету "Управление проектной документацией". Он может быть использован для создания документа, описывающего процесс реализации проектной документации в организации.

% Описание процесса
\section{Описание процесса}

\subsection{Цель}
Цель процесса реализации по предмету "Управление проектной документацией" заключается в обеспечении эффективного создания, управления и хранения проектной документации.

\subsection{Шаги процесса}
Процесс реализации может включать следующие шаги:

\begin{enumerate}
    \item Определение требований к документации проекта.
    \item Планирование создания и обновления документации.
    \item Составление и форматирование документов.
    \item Рецензирование и утверждение документов.
    \item Распространение и контроль версий документов.
    \item Архивирование и хранение документов.
\end{enumerate}

% Инструменты и методы
\section{Инструменты и методы}
В процессе реализации могут использоваться следующие инструменты и методы:

\begin{itemize}
    \item Система управления документами (например, SharePoint, Google Диск).
    \item Инструменты для создания и форматирования документов (например, Microsoft Word, LaTeX).
    \item Методы рецензирования и утверждения документов (например, совещания, электронные системы комментирования).
    \item Инструменты контроля версий (например, Git, SVN).
\end{itemize}

% Заключение
\section{Заключение}
Данный шаблон предоставляет общую структуру для процесса реализации по предмету "Управление проектной документацией". Он может быть адаптирован и расширен в соответствии с требованиями вашей организации.

\end{document}